
\noindent
\setlength{\parindent}{0pt}
\begin{spacing}{1.5}


\section*{Abstract}
The 'Sternsinger Aktion' is a customary fundraising event in which children perform door-to-door while soliciting donations. In Lieboch, it traditionally relied on manual, paper-based planning. While functional, it lacked the convenience and adaptability of digital solutions. This diploma thesis aimed to improve the process by digitalizing the planning and execution, making it more efficient and manageable.\blankLine


Three tools were developed: an Admin Panel for address assignment, a mobile app for participants to view their assigned addresses, and a software component for managing data exchange between the applications and the database. The software adapts to changes, such as last-minute adjustments to group assignments.\blankLine


This thesis introduces the technologies used and key features of the tools, including adaptive algorithms and real-time data integration, area border consideration for group assignments, and a focus on usability. Instructions for using the tools are also provided.\blankLine


The developed solution was successfully tested with four participants in Lieboch. Overall, the digitalization of the 'Sternsinger Aktion' was successful. This diploma thesis presents a reliable system that highlights the advantages of digital tools in event planning and lays a solid foundation for future improvements and expansion.\blankLine


\newpage

\section*{Zusammenfassung}
Die Sternsingeraktion ist eine traditionelle Spendenaktion, bei der Kinder singend von Tür zu Tür gehen und Spenden sammeln. In Lieboch basierte sie traditionell auf einer manuellen, papierbasierten Planung. Diese war zwar funktional, bot aber nicht den Komfort und die Anpassungsfähigkeit von digitalen Lösungen. Ziel dieser Diplomarbeit war es, den Prozess zu verbessern, indem die Planung und Durchführung digitalisiert und damit effizienter und überschaubarer wird.\blankLine

Es wurden drei Tools entwickelt: ein Admin-Panel für die Adresszuweisung, eine mobile App, mit der die Teilnehmer ihre zugewiesenen Adressen einsehen können, und eine Softwarekomponente für die Verwaltung des Datenaustauschs zwischen den Anwendungen und der Datenbank. Die Software passt sich an Änderungen an, so wie kurzfristige Anpassungen der Gruppenzuweisungen.\blankLine

In dieser Arbeit werden die verwendeten Technologien und die wichtigsten Merkmale der Tools vorgestellt, darunter adaptive Algorithmen und Datenintegration in Echtzeit, die Berücksichtigung von Gebietsgrenzen bei der Gruppenzuweisung und ein Schwerpunkt auf der Benutzerfreundlichkeit. Darüber hinaus werden Anleitungen zur Nutzung der Tools bereitgestellt.\blankLine

Die entwickelte Lösung wurde mit vier Teilnehmern in Lieboch erfolgreich getestet. Insgesamt war die Digitalisierung der Sternsingeraktion erfolgreich. Mit dieser Diplomarbeit wird ein zuverlässiges System vorgestellt, das die Vorteile digitaler Tools in der Veranstaltungsplanung aufzeigt und eine solide Grundlage für zukünftige Verbesserungen und Erweiterungen schafft.\blankLine


\end{spacing}
