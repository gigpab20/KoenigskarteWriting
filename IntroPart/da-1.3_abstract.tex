
\noindent
\setlength{\parindent}{0pt}
\begin{spacing}{1.5}


\section*{Abstract}
The Sternsinger Aktion in Lieboch traditionally used manual, paper-based planning. While functional, it lacked the convenience and adaptability of digital solutions. This thesis aimed to improve the process by digitalizing the planning and execution, making it more efficient and manageable.\\

Three tools were developed: an Admin Panel for address assignment, a mobile app for participants to view their assigned addresses, and a software component for managing data exchange between the applications and the database. The software adapts to changes, such as last-minute adjustments to group assignments.\\

This thesis introduces the technologies used and key features of the tools, including adaptive algorithms, area border consideration for group assignments, and a focus on usability. Instructions for using the tools are also provided.\\

The software was successfully tested with four groups in Lieboch. Overall, the digitalization of the Sternsinger Aktion was successful. The software created in this thesis offers a reliable solution, showcasing the benefits of digital tools in event planning and will be used and developed further.\\

\newpage

\section*{Zusammenfassung}
Die Sternsinger-Aktion in Lieboch verwendete traditionell eine manuelle, papierbasierte Planung. Diese war zwar funktional, bot aber nicht den Komfort und die Anpassungsfähigkeit digitaler Lösungen. Ziel dieser Arbeit war es, den Prozess zu verbessern, indem die Planung und Durchführung digitalisiert und damit effizienter und überschaubarer gemacht wird.\\

Es wurden drei Tools entwickelt: ein Admin-Panel für die Adresszuweisung, eine mobile App, mit der die Teilnehmer ihre zugewiesenen Adressen einsehen können, und eine Softwarekomponente zur Verwaltung des Datenaustauschs zwischen den Anwendungen und der Datenbank. Die Software passt sich an Änderungen an, wie Anpassungen der Gruppenzuweisungen in letzter Minute.\\

In dieser Arbeit werden die verwendeten Technologien und die wichtigsten Merkmale der Tools vorgestellt, darunter adaptive Algorithmen, die Berücksichtigung von Bereichsgrenzen bei Gruppenzuweisungen und ein Schwerpunkt auf der Benutzerfreundlichkeit. Außerdem werden Anleitungen zur Nutzung der Tools gegeben.\\

Die Software wurde mit vier Gruppen in Lieboch erfolgreich getestet. Insgesamt war die Digitalisierung der Sternsingeraktion erfolgreich. Die in dieser Arbeit erstellte Software bietet eine verlässliche Lösung, die die Vorteile digitaler Tools in der Veranstaltungsplanung aufzeigt und weiter genutzt und entwickelt werden soll.\\

\end{spacing}
