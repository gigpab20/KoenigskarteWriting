% Englische Sprache
\usepackage[english]{babel}
% Deutsche Umlaute
\usepackage[utf8]{inputenc}
\usepackage[T1]{fontenc}
% Höhe der Kopf- und Fußzeile definieren
\usepackage{geometry}
\geometry{verbose,a4paper,tmargin=25mm,bmargin=25mm,lmargin=25mm,rmargin=25mm, headsep=0.5cm}
% Weitere Packages
\usepackage[titletoc]{appendix}
\usepackage{csquotes}
\usepackage{blindtext}
\usepackage{setspace} 
\usepackage[table]{xcolor}
\usepackage{xcolor}
\usepackage{graphicx}
\usepackage{tikz}
\usepackage{ifthen}
\usepackage[]{ragged2e}
\usepackage{lmodern}
\usepackage{fancyhdr}
\usepackage{xcolor,colortbl}
\usepackage{hyperref}
\usepackage{titlesec}
\setlength\parindent{0pt}
\usepackage[font={footnotesize}]{caption}
% imports für tabellen
\usepackage{longtable}
\usepackage{multirow}
\usepackage{tabularx}

% SI-Einheiten
% from texlive-science
\usepackage{siunitx}
\sisetup{locale = DE,per-mode=fraction}

% Literaturvereichnis / Zitate
\usepackage[
    backend=biber,
    style=numeric,     %apa or numeric
    autocite=inline]{biblatex}
    

% Links im Inhaltsverzeichnis
\hypersetup{
    colorlinks,
    citecolor=black,
    filecolor=black,
    linkcolor=black,
    urlcolor=black
}
% Schriftart einbinden
\usepackage{sourcesanspro} % roman is default font
\usepackage{courier} % used for \code{arg1}  \inlinecode

% Inhaltsverzeichnis umbennen
\addto\captionsenglish{
  \renewcommand{\contentsname}
    {Table of Contents}
}

\addto\captionsngerman{
  \renewcommand{\contentsname}
    {Table of Contents}
}

% Literaturverzeichnis
%\usepackage[numbers]{natbib} 

% Caption-Abkürzungen
\addto\captionsenglish{\renewcommand{\figurename}{Fig.}}
\addto\captionsngerman{\renewcommand{\figurename}{Abb.}}

% Fußnote
\usepackage[bottom]{footmisc}

\include{DA-INCLUDES/da-syntaxHighlighting.tex}

% Bilder floaten
\usepackage{float}
\usepackage{wrapfig}
\newfloat{listing}{H}{loc}
\floatname{listing}{Listing}

% Zeilenumbruch in Tabelle
\usepackage{makecell}

% Blocksatz
\usepackage{microtype}

\setcounter{secnumdepth}{4}

\titleformat{\paragraph}
{\normalfont\normalsize\bfseries}{\theparagraph}{0.7em}{}
\titlespacing*{\paragraph}
{0pt}{3.25ex plus 1ex minus .2ex}{1.5ex plus .2ex}

\pagestyle{fancy}

% Schriftart festlegen
\renewcommand{\familydefault}{\rmdefault}  % Serifenschrift
%\renewcommand{\familydefault}{\sfdefault}  % Serifenlose Schrift

% Text in Kopfzeile für Vorwort festlegen
\renewcommand{\sectionmark}[1]{\markright{\thesection\ #1}}

% Fußzeilenlinie definieren
\renewcommand{\footrulewidth}{0.4pt}

% Autor-Kommando definieren
\newcommand{\TheAuthor}{}
\newcommand{\Author}[1]{\renewcommand{\TheAuthor}{#1}}

% Zellenhöhe für Tabelle
\renewcommand{\arraystretch}{1.3}

% Inline-Code
\newcommand{\inlinecode}{\texttt}
\let\oldtexttt\texttt
\newcommand{\code}[1]{\colorbox{lightgray!25}{\oldtexttt{#1}}} 
\renewcommand{\texttt}{\code}

\newenvironment{longcite}
{
 	\begin{quote}
 		\small
		\itshape
			}
			{	
	\end{quote}
}


\newcommand*{\inputLanguageText}[1]{
	\input{IntroPart/#1}
}

\newcommand*{\inputDepartmentTitlePage}[1]{
	\input{IntroPart/da-1.0_titlePage-en.tex}
}


