%Beispiele für Code-Snippets
%C#-Code
%\begin{center}
%\lstset{style=mycsharp, caption=CSharp example}
%\begin{lstlisting}
%public class Class : IList
%{
%	public object Test()
%	{
%		// return object
%		return null;
%	}
%}
%\end{lstlisting}
%\end{center}

%Python-Code
%\begin{center}
%\lstset{style=mypython, caption=Python example}
%\begin{lstlisting}
%def f(x):
%    return x
%\end{lstlisting}
%\end{center}

%C-Code
%\begin{center}
%\lstset{style=myc, caption=C example}
%\begin{lstlisting}
%#include <stdio.h>
%int main(void)
%{
%   printf("Hello World!"); 
%}
%\end{lstlisting}
%\end{center}

%XML-Code
%\begin{center}
%\lstset{style=myxml, caption=XML example}
%\begin{lstlisting}
%<?xml version="1.0" encoding="utf-8"?>
%<xs:schema attributeFormDefault="unqualified" elementFormDefault="qualified"
%   xmlns:xs="http://www.w3.org/2001/XMLSchema">
%  <xs:element name="points">
%    <xs:complexType>
%      <xs:sequence>
%        <xs:element maxOccurs="unbounded" name="point">
%          <xs:complexType>
%            <xs:attribute name="x" type="xs:unsignedShort" use="required" />
%            <xs:attribute name="y" type="xs:unsignedShort" use="required" />
%          </xs:complexType>
%        </xs:element>
%      </xs:sequence>
%    </xs:complexType>
%  </xs:element>
%</xs:schema>
%\end{lstlisting}
%\end{center}
