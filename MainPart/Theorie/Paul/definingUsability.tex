\author{\daAuthorTwo}

Since my research question "How can user-experience principals add to an intuitive map displayment for nonprofit activities in which people of different technical know-how levels collaborate?" is all about usability, I want to introduce you to its basic concepts and challenges but also provide some examples on how usability can impact a software's revenue and perception.

\blankLine

Usability is a critical aspect of software and interface design, ensuring that users can efficiently and effectively interact with a product or system. Its job is to provide clear feedback and "experiences" to the user, so interactions between software and human feel smooth and straight forward. Because each human being is different in its emotional experiences, it is difficult to design a kind of "one size fits all" solution. Due to this circumstance, many studies and experiments were conducted.   
\autocite{Paul:Usability101}

2.2 Fundamental Concepts of Usability

Usability principles are grounded in research and best practices. According to Schneiderman (1997), usability in human-computer interaction (HCI) relies on consistency, visibility, feedback, and simplicity. Consistency ensures users do not need to learn new interactions for each task. Visibility allows users to understand their options at any given moment, while feedback provides them with immediate confirmation of their actions. Simplicity, on the other hand, minimizes cognitive load and prevents unnecessary complications (Schneiderman, 1997).

2.3 Challenges in Designing for a Broad User Spectrum

Designing for a diverse user base requires addressing varying levels of experience, cognitive abilities, and accessibility needs. A study on user experience principles highlights that inclusive design considers users with disabilities, non-native language speakers, and individuals with varying levels of digital literacy (Nnesirr, 2023). Failure to account for these differences can lead to usability issues, preventing certain groups from effectively using a system. Designers must implement features such as adjustable text sizes, screen reader compatibility, and intuitive navigation to ensure accessibility for all users (Nnesirr, 2023).

3. Usability in the Context of Maps

3.1 Basic Analysis of the Google Maps Interface

Google Maps is one of the most widely used navigation tools, offering a range of features such as real-time traffic updates, route planning, and location discovery. Its interface follows usability principles by prioritizing clarity and responsiveness. The search function is prominently displayed, while key actions, such as zooming and switching between map views, are easily accessible (Nielsen, 2012). Additionally, Google Maps leverages user feedback mechanisms, such as reporting incorrect locations, to enhance accuracy and user experience.

3.2 Identifying Flaws in Google's Design

Despite its strengths, Google Maps has usability shortcomings. Research has shown that the interface can be overwhelming due to the excessive number of icons and options displayed simultaneously (Schneiderman, 1997). Users, particularly those unfamiliar with digital tools, may struggle with distinguishing between essential and non-essential features. Additionally, the reliance on small touch targets can make navigation difficult for users with motor impairments (Nnesirr, 2023). Addressing these flaws would enhance accessibility and usability for a broader audience.

3.3 How Could Specific User Groups Struggle With This Design?

Certain user groups face greater challenges when using Google Maps. For example, older adults may find it difficult to read small text labels or interpret color-coded information. Individuals with visual impairments may struggle with low-contrast elements and a lack of alternative text for key icons (Nnesirr, 2023). Moreover, users with cognitive disabilities may become overwhelmed by the interface’s complexity, necessitating a more simplified mode tailored to their needs. Incorporating user-centered design principles, such as progressive disclosure and customizable settings, could alleviate these issues and create a more inclusive experience.

4. Conclusion

Usability is essential for creating intuitive digital experiences, particularly for applications like Google Maps that serve a wide user base. By adhering to fundamental usability principles—such as consistency, feedback, and accessibility—designers can enhance user satisfaction and efficiency. However, challenges remain in designing for diverse user groups. Addressing these issues requires ongoing evaluation and adaptation, ensuring that all users, regardless of their technical proficiency or abilities, can navigate digital tools with ease.


\subsection{Why it is important}

Usability ensures that users can accomplish their goals with minimal frustration and maximum efficiency. With the increasing reliance on digital tools, usability plays a key role, not only, in shaping user experiences but also accessibility of software for diverse user groups. A well-designed and thought-out usability concept can go a long way from refining a once tedious and complicated to use product, to one that can be operated even by non-familiar users or disabled people. This plays a big part in the inclusion of all age and knowledge groups.

\subsection{Components of Usability}

According to Jakob Nielsen, usability consists of five core components:

\begin{itemize}
    \item Learnability
    \begin{description}
        \item How \textbf{easy} it is to accomplish basic tasks the first time  
    \end{description}
    
    \item Efficiency
    \begin{description}
        \item How \textbf{quickly} task can be accomplished after an initial learning period
    \end{description}

    \item Memorability
    \item Error handling
    \item Satisfaction
\end{itemize}
Learnability refers to how quickly new users can understand a system, while efficiency focuses on task completion speed. Memorability ensures that users can return to a system without relearning it, and effective error handling reduces user frustration. Ultimately, usability leads to higher user satisfaction and increased adoption rates (Nielsen, 2012).

\subsection{Fundamental concepts}

\subsection{Challenges in designing for a broad user spectrum}

\newpage