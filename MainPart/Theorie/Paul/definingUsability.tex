\author{\daAuthorTwo}

Since my research question "How can user-experience principals add to an intuitive map displayment for nonprofit activities in which people of different technical know-how levels collaborate?" is all about usability, I want to introduce you to its basic concepts and challenges but also provide some examples on how usability can impact a software's revenue and perception.

\blankLine

Usability is a critical aspect of software and interface design, ensuring that users can efficiently and effectively interact with a product or system. Its job is to provide clear feedback and "experiences" to the user, so interactions between software and human feel smooth and straight forward. Because each human being is different in its emotional experiences, it is difficult to design a kind of "one size fits all" solution. Due to this circumstance, many studies and experiments were conducted.   
\autocite{Paul:Usability101}



\subsection{Why it is important}

Usability ensures that users can accomplish their goals with minimal frustration and maximum efficiency. With the increasing reliance on digital tools, usability plays a key role, not only, in shaping user experiences but also accessibility of software for diverse user groups. A well-designed and thought-out usability concept can go a long way from refining a once tedious and complicated to use product, to one that can be operated even by non-familiar users or disabled people. This plays a big part in the inclusion of all age and knowledge groups as well as the general market share through mass adoption because of the easiness.

\subsection{Components of Usability}

According to Jakob Nielsen, usability consists of five core components. To achieve the best possible usability, each of factors must be taken into account and be improved to its maximum.

\begin{itemize}
    \item Learnability
    \begin{description}
        \item How \textbf{easy} it is to accomplish basic tasks the first time  
    \end{description}

    \item Efficiency
    \begin{description}
        \item How \textbf{quickly} task can be accomplished after an initial learning period
    \end{description}

    \item Memorability
    \begin{description}
        \item How \textbf{memorable} actions are to users so, after an extended period of not using a software, how good have they memorized  the process
    \end{description}

    \item Error handling
    \begin{description}
        \item How \textbf{many} errors users make while using the design and how \textbf{sever} they are  
    \end{description}

    \item Satisfaction
    \begin{description}
        \item How \textbf{pleasant} the overall experience of using the product is 
    \end{description}
\end{itemize}
\autocite{Paul:Usability101}

\blankLine

Now that we know these key points, what measures can we take to reach the goal of great usability? According to Nasrullah Hamidli, human-computer-interaction relies on consistency, visibility, feedback, and simplicity. Consistency ensures users do not need to learn new interactions for each task. For example, buttons should look alike and be in a similar location. This makes for a more natural navigation across the product and an overall familiar feel. Simplicity connects directly to this. Its goal is to minimize clutter and make user interfaces easy to understand and provide one, clear way to accomplish a task, not many possible, but complicated and unintuitive ways. It also aims to reduce distractions. Visibility allows users to clearly understand their options at any given moment, this is most often achieved through visual cues, like, grayed out buttons. This goes hand in hand with the feedback aspect, which provides immediate confirmation of actions. Loading indicators, color-changes and alike get used most often.

Another important part of designing a good UI are typography and colors. These
\todo{mir fallt kein fucking wort ein} can influence the attention and emotions of users, as well as establish visual hierarchies, which, intern, contribute again to a simpler to navigate interface. 

\autocite{Paul:UIUXIntroduction}

\subsection{Fundamental concepts}

\subsection{Challenges in designing for a broad user spectrum}

Designing for a diverse user base requires the addressing of varying levels of experience, prior knowledge, cognitive abilities, and accessibility needs. Failure to account for these differences can lead to usability issues, preventing certain groups from effectively using a system. Designers must implement features such as adjustable text sizes, screen reader compatibility, and intuitive navigation to ensure accessibility for all users.
\autocite{Paul:UIUXIntroduction}


\newpage