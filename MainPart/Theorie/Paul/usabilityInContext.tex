\Author{\daAuthorTwo}

In this section we will further inspect the usability of mapping solutions like Google Maps. We will identify some flaws of Google's design choices and how they could influence specific user groups.

\blankLine

\subsection{Basic Analysis of the Google Maps Interface}

Google Maps is one of the most widely used mapping and navigation applications globally. Its feature set includes real-time traffic updates, route planning, and location discovery. Generally speaking, it is quite difficult to design a simultaneously user-friendly and functional mapping application. Maps get overloaded and confusing quite easily. They are bloated with information like street names, house numbers, borders, and geographical features like rivers or lakes. Due to this fact, maps are not easy to design according to the principles of usability. But Google developed a very good and intuitive concept on which we now will take a look.

\blankLine

The interface of Google Maps consists of a map, search bar, and a menu for additional settings. The search function is prominently displayed as it is the most common tool used. This is a good use of \textit{visual hierarchy} as the initial focus when opening the app immediately gets drawn to the bigger hint text in the search bar on top. Below it, there is a list of buttons, so users can quickly search for local places that match a specific category, like restaurants, cafés, or gas stations. Notably, there are no buttons for movement actions such as zooming or panning; all this is controlled through swipe and pinch gestures directly on the map. \autocite{battersby2008user}

\blankLine

To start navigating to a specific area, you can search for the street name and house number, or the name of a company or other details. Maps gives you recommendations and tries to provide auto-suggestions for your target. When the user selects a destination, the navigation can be started through a big blue button. This is an example of applied \textit{color theory}. The route gets marked by an again bright blue line, which creates a good contrast to the other colors used on the map and captures the attention of the user. This line also has multiple purposes other than displaying the route; for example, if a part is orange, that means the traffic at this point is beginning to jam. If then there is a full stop traffic jam, the line turns red at this section. Also, icons for speed cameras or accidents that other users reported get displayed along it.

\blankLine

One of the core strengths of Google Maps is its interactive and responsive design. Users can zoom in and out using intuitive pinch gestures on mobile devices or scroll actions on desktops. The transition between zoom levels is smooth, preserving context and avoiding disorientation through too big scaling steps. Additionally, the map changes depending on the zoom level. Street names and buildings get displayed only if the user has zoomed in enough. The same happens with markers for businesses and other map data. Through this concept, Google ensures that users do not get overwhelmed. \autocite{battersby2008user}

\blankLine

\subsection{Identifying Flaws in Google's Design}

While Maps is a well-polished product, it is not perfect. One significant flaw is the cognitive overload caused by excessive information. The inclusion of business listings, other suggested routes, live traffic data, and user-generated content can make it difficult for users to focus on their primary navigation tasks. \autocite{battersby2008user}

\blankLine

\subsection{How Could Specific User Groups Struggle with This Design}

Google Maps caters to a broad spectrum of users, but its design can pose difficulties for certain demographics:
\blankLine

\textbf{Elderly Users:} Many elderly individuals may find the interface overwhelming due to small text sizes, densely packed information, and complex menus. Their unfamiliarity with modern digital navigation tools may lead to confusion, especially when trying to search for locations or adjust route preferences. A lack of prominent, simplified navigation options tailored to this group amplifies the issue. \autocite{allyant2022google}

\textbf{Users with Low Digital Literacy:} People who are not well-versed in digital technology could struggle with Google Maps' multitude of features. They may have difficulty understanding icons, switching between different map modes, or using advanced functionalities like saved locations and street view. A more guided and \textit{simplified} mode could enhance their experience.

\textbf{Users with Disabilities:} Visually impaired users may struggle with \textit{insufficient contrast}, small icons, and the \textit{lack of tactile feedback}. While screen readers can assist, Google Maps does not always provide clear, structured data for these tools. Additionally, users with motor impairments may find it hard to navigate menus and interact with small buttons, particularly on touch screens. \autocite{froehlich2019grand}

\blankLine

By addressing these usability concerns, Google Maps could enhance its interface to be more intuitive and accessible for diverse user groups.