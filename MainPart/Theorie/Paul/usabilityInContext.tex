\subsection{Basic Analysis of the Google Maps interface}

\subsection{Identifying Flaws in Googles Design}

\subsection{How could specific user groups struggle with this design}


3. Usability in the Context of Maps

3.1 Basic Analysis of the Google Maps Interface

Google Maps is one of the most widely used navigation tools, offering a range of features such as real-time traffic updates, route planning, and location discovery. Its interface follows usability principles by prioritizing clarity and responsiveness. The search function is prominently displayed, while key actions, such as zooming and switching between map views, are easily accessible (Nielsen, 2012). Additionally, Google Maps leverages user feedback mechanisms, such as reporting incorrect locations, to enhance accuracy and user experience.

3.2 Identifying Flaws in Google's Design

Despite its strengths, Google Maps has usability shortcomings. Research has shown that the interface can be overwhelming due to the excessive number of icons and options displayed simultaneously (Schneiderman, 1997). Users, particularly those unfamiliar with digital tools, may struggle with distinguishing between essential and non-essential features. Additionally, the reliance on small touch targets can make navigation difficult for users with motor impairments (Nnesirr, 2023). Addressing these flaws would enhance accessibility and usability for a broader audience.

3.3 How Could Specific User Groups Struggle With This Design?

Certain user groups face greater challenges when using Google Maps. For example, older adults may find it difficult to read small text labels or interpret color-coded information. Individuals with visual impairments may struggle with low-contrast elements and a lack of alternative text for key icons (Nnesirr, 2023). Moreover, users with cognitive disabilities may become overwhelmed by the interface’s complexity, necessitating a more simplified mode tailored to their needs. Incorporating user-centered design principles, such as progressive disclosure and customizable settings, could alleviate these issues and create a more inclusive experience.

4. Conclusion

Usability is essential for creating intuitive digital experiences, particularly for applications like Google Maps that serve a wide user base. By adhering to fundamental usability principles—such as consistency, feedback, and accessibility—designers can enhance user satisfaction and efficiency. However, challenges remain in designing for diverse user groups. Addressing these issues requires ongoing evaluation and adaptation, ensuring that all users, regardless of their technical proficiency or abilities, can navigate digital tools with ease.


\newpage