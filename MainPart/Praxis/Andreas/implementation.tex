%! Author = weiss
%! Date = 20.01.2025
\Author{\daAuthorThree}

    This program's backend which offers a reliable and scalable system for handling all addresses, streets, areas and special features, is the essential layer that makes sure the admin panel and the mobile app for the guides run well. \newline
    This backend serves as the center for data processing and communication which supports the admin and all guides through an unified API. \newline
    It effectively manages all requests from the mobile app, where guides interact with addresses which are in their area. Furthermore, it manages all necesssary requests from the admin panel which enables accurate management and control over anything relevant for the guides such as addresses, areas, streets and special features an address may has. \newline 

    One big function of the backend is to enable the admins to perform a wide range on CRUD (Create, Read, Update, Deleter) operations on all necesssary data. Nevertheless, the other function of the backend is to provide guides with their relevant area and all the addresses in this area. This functionalities are crucial for managing the caroler campaign by the admin and to ensure that the guides have access in the mobile app to accurate and up-to-date addresses and areas. \newline

    We use Sprin Boot in our Backend which ensures seamless data transfer between the user interfaces (admin panel and mobile app) and the database of this project. By having an organized structure with the different layers (configuration, entities, repositories, services and controllers), we ensure that the system remains flexible, scalable and easy to maintain. \newline 

    Now, this chapter will show the backend's architecture and therefore all its components while also describing how each component functions and contributes to the whole system of the backend.

    \subsection{Config of Spring Boot (application.properties)}
    A Spring Boot' \texttt{applications.properties} file is crucial for specifying different setting that control the program's behavior. It contain settings for server parameters, logging, database connections and security elements which ensure that the application is running efficiently and securely. This \texttt{applications.properties} file is the reason why Spring Boot is one of the most easy frameworks to configure.

    \subsubsection{Database Configuration}
    This section defines the connection settings to the PostgreSQL database. In our case this details are set to this: 
    \lstset{style=mycsharp, caption=Database Configuration}
        \begin{lstlisting}
spring.datasource.driver-class-name=org.postgresql.Driver
spring.datasource.url=jdbc:postgresql://localhost:5432/postgres
spring.datasource.username=postgres
spring.datasource.password=postgres
        \end{lstlisting}
    In this case the first code line defines the driver that is used to interact with the PostgreSQL database. \newline
    Similarly, the second code line provides the URL where the database is hosted. This time, the database is running localy and on the default PostgreSQL port \texttt{5432}, furthermore, the database name is set to \texttt{postgres}. \newline
    The username and password set in the properties are both \texttt{postgres} through the last two code lines. It is important to note that these settings are for local testing and development. In a production environment, these properties would need to be replaced with more secure settings. \newline

    This file also configures the Java Persistence API (JPA) and Hibernate which control the database interface, in addition to the database settings. In our case this settings look like this:
    \lstset{style=mycsharp, caption=JPA Configuration}
        \begin{lstlisting}
spring.jpa.show-sql=true
spring.jpa.properties.hibernate.format_sql=true
spring.jpa.hibernate.ddl-auto=none
spring.jpa.properties.hibernate.dialect=org.hibernate.dialect.PostgreSQLDialect
        \end{lstlisting}
    The first code line enables logging of SQL queries which get executed by Hibernate. This helps in debugging and verifying that the operations are being performed on the database. \newline
    The next code line only ensures that the SQL queries get formatted right for better readability in the logs. \newline
    The third code line prevents Hibernate from automatically modifying the database structure to avoid accidental changes in the tables of the database. \newline
    The last line is a setting to ensure that Hibernate generates SQL statements that comply with the PostgreSQL rules so that the SQL statements are compatible with our PostgreSQL database.


    \subsection{Entity Classes (Structure/Purpose)}
    Entity classes define the application's data model, using annotations to map fields to database tables.

    \subsection{JPA-Repositories (DB Access and CRUD Operations)}
    Repositories simplify database access by providing methods for CRUD operations and enabling custom queries.

    \subsection{Service Classes}
    Service classes encapsulate business logic, coordinating data flow between controllers and repositories.

    \subsection{Rest Controller (API Endpoints and their Functions)}
    REST controllers define API endpoints, processing requests and returning responses to ensure seamless interaction with the frontend.
