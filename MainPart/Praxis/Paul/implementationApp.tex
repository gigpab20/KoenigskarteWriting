\Author{\daAuthorTwo}

In this section of our thesis, the functional implementation of the mobile app will be described. We will show how the FlutterMap component is build and what logic works in the background.

\subsection{Address-Provider}

Since without data, our app would be useless, we will start with describing the Address-Provider class. It is implemented using the \texttt{provider-package} by flutter. With its help, we can easily notify all screens and widgets, when the data was updated.

\blankLine

The main objective of the Address-Provider is, to communicate with the backend and GraphHopper instance, but also so bring the received data into the correct format for our frontend. For this we defined four different instance variables. A list, that stores all addresses, one that only holds the coordinates of the \texttt{border addresses} returned by the backend, a map that is used for coordinates of the routing points for a specific street (\textit{Map<String[street name]>, List<LatLng>[coordinates]}) and finally a map for storing the addresses grouped by their street. Every component that owns an instance of this Address-Provider can access these variables which makes it easy to display the data.

\blankLine

Due to the fact that, the backend returns the addresses unsorted, the Address-Provider also takes on this task. For the grouping by the street name, we can just use the map from before, but to then also sort the addresses ascending by their house number, we needed to write a function to compare them.

\blankLine

The \textit{compareHouseNumbers} function enables natural sorting of house numbers by separating numeric and non-numeric parts. It extracts sequences of digits and letters, then compares numeric parts as integers and non-numeric parts alphabetically. This ensures that, for example, "10a" is correctly sorted before "10c". The function is useful for maintaining an intuitive order in the address list, improving readability and usability.

\lstset{style=generic, caption=compareHouseNumbers function (AddressProvider.dart)}
\begin{lstlisting}
    int compareHouseNumbers(String a, String b) {
        final regex = RegExp(r'(\d+|\D+)');
    
        final partsA = regex.allMatches(a).map((match) => match.group(0)!).toList();
        final partsB = regex.allMatches(b).map((match) => match.group(0)!).toList();
    
        for (int i = 0; i < partsA.length && i < partsB.length; i++) {
          final partA = partsA[i];
          final partB = partsB[i];
    
          //compare numeric part
          if (int.tryParse(partA) != null && int.tryParse(partB) != null) {
            final numA = int.parse(partA);
            final numB = int.parse(partB);
            if (numA != numB) return numA.compareTo(numB);
          } else {
            // Compare alphabetic part
            final comparison = partA.compareTo(partB);
            if (comparison != 0) return comparison;
          }
        }
    
        return partsA.length.compareTo(partsB.length);
      }
\end{lstlisting}

\subsection{HTTP-Requests}

To actually receive and the data in our database, we built our backend API. It can be accessed through different HTTP-Requests. The mobile application makes use of the following routes: 

\begin{itemize}
    \item \textbf{fetch all addresses of area}
    
    As the name suggests, fetches and saves all addresses from the area that is set through the QR-Code. Here the \textit{compareHouseNumbers} function gets used.

    \lstset{style=generic, caption=use of compareHouseNumbers function (AddressProvider.dart)}
    \begin{lstlisting}
        addressMap = groupBy(addresses, (Address address) => address.street.name);

        addressMap = addressMap.map((key, value) {
          value.sort((a, b) => compareHouseNumbers(a.houseNumber, b.houseNumber));
          return MapEntry(key, value);
        });
    \end{lstlisting}

    \item \textbf{fetch border addresses of area}
    Fetches and extracts the coordinates of the border addresses returned by the backend.

    \lstset{style=generic, caption=extraction of coordinates from border addresses (AddressProvider.dart)}
    \begin{lstlisting}
        borderAddresses = decodedJSON.map((json) => Address.fromJson(json)).toList().map((point) => LatLng(point.latitude, point.longitude)).toList();
    \end{lstlisting}

    \item \textbf{toggle address visited state}
    
    Used by the list screen, on swipe of an individual address-item.

    \item \textbf{toggle street visited state}
    Virtually the same as toggle address, but now for the whole street. Gets called when a user swipes and confirms the change on a whole street.

    \item \textbf{update comment of address}
    
    Used by the \textit{details pop-up} on save with a new comment in the text field.

    \item \textbf{fetch route information for streets}
    
    Call to our GraphHopper instance. Iterates over every key of the \textit{addressMap}, requests the routing data and calls function \textit{extractPoints} to extract only valid data out of the whole JSON response.
    
    \todo{vielleicht noch die extractPoints function da rein?}
\end{itemize}

\subsection{Implementation of the FlutterMap Component}

The FlutterMap widget is build up in layers, which all can be individually ordered and configured. In this part, the layers that we utilized and what configuration options we set will be described. 

\begin{itemize}
  \item \textbf{Tile-Layer}
  
  This layer imports the tiles, so, the actual map, of OpenStreetMaps' tile server. In future iterations, a different server or tile style (vector tiles) could be used to improve performance and usability further.

  \item \textbf{Polygon-Layer}
  
  In this layer, we draw the polygon that marks the border of the selected area. To achieve this, we just create a new Polygon object with the coordinates saved in the \textit{borderAddresses} list of the \textit{AddressProvider} and set a few styling options, like the stroke width and border color.

  \item \textbf{Polyline-Layer}
  
  This layer was utilized to display the streets that correspond to the specified area. Here we draw different poly lines based on the routing data we fetched from our GraphHopper server. In the future, this feature could be further improved so, different streets get drawn in varying colors. For now, every street poly line gets displayed in blue. 

  \item \textbf{Marker-Layer}
  
  This is the final and arguably most important layer, because here we set a marker for every address. This marker can vary, depending on the specification that the admin assigned to an address in the Admin-Panel. This was implemented by decoding the Base64 string, that is saved in the address object and stores the marker-image. Additionally, if a comment is saved in the address, a little warning indicator gets display on the upper right-hand side of this encoded image. This got introduced, so that users could see at a glance if there was something special to know about an address. We solved this by using a Stack widget. Additionally, we added a GestureDetector, on top of this stack. This is used to open the details' dialog in which the user can modify the comment field. 
\end{itemize}

\newpage