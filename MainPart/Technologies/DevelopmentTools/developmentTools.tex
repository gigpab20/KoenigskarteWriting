\Author{\daAuthorTwo}

\subsubsection{VS Code}
Visual Studio Code (VS Code) is a version of the Code - OSS project with Microsoft-specific customizations included which makes it a closed-source product licensed by Microsoft. VS Code is highly customizable, with support for community made plugins and many opportunities for the user to define how things should act and look. \autocite{VSCode:readMe} There are also many handy features implemented out of the box, like native VCS support. Due to the fact VS Code is an electron based application, it can not only be used on desktop apps but also directly in the browser of any device. \autocite{VSCode:electron} \autocite{VSCode:web}

\begin{figure} [H]
    \center
    \includegraphics [width=0.3\textwidth] {images/Technologies/vscodeLogo.png}
    \caption{VS Code Logo (Source: \url{https://code.visualstudio.com/brand})}
\end{figure}

\subsubsection{IntelliJ}
IntelliJ IDEA is an IDE primarily designed for Java and Kotlin developers, published by JetBrains. In comparison to VS Code, it is much heavier with the reason being the included tools like the direct integration of a database connection via the embedded DataGrip version. It is the base framework used for other, more specific IDEs by JetBrains, like Android Studio, WebStorm or DataGrip. Its functionality can also be extended by making use of plugins. 
\autocite{IntelliJ:JetBrains}
\autocite{IntelliJ:Wikipedia}

\begin{figure} [H]
    \center
    \includegraphics [width=0.4\textwidth] {images/Technologies/intellijLogo.png}
    \caption{IntelliJ IDEA Logo (Source: \url{https://www.jetbrains.com/company/brand/})}
\end{figure}

\subsubsection{Android Studio}
Android Studio is also an IDE made by JetBrains. As mentioned before, it is based on the IntelliJ framework but, as the name suggests, with the specification to developing Android apps. Aside from Java and Kotlin based, pure Android applications, it can also be used for developing cross-platform applications using the flutter framework. The open-source IDE introduced by Google also features an Emulator for Pixel devices which simplifies testing. Last but not least it provides an intuitive way to compile the app and flash it to a physical Android device using ADB and USB-debugging. 
\todo{logo suchen}

\subsubsection{Postman}
Postman is an API platform, mainly used to test APIs.  It features collaboration and an automatic code-generation, so you can design the request you want to make in postman and export it in the language you use after. We used it to develop our API, without a frontend application to minimize the points of failure.

\begin{figure} [H]
    \center
    \includegraphics [width=0.4\textwidth] {images/Technologies/postmanLogo.png}
    \caption{Postman Logo (Source: \url{https://www.postman.com/legal/logo-usage/})}
\end{figure}


\subsubsection{Figma}
Lorem ipsum dolor sit amet, consetetur sadipscing elitr, sed diam nonumy eirmod tempor invidunt ut labore et dolore magna aliquyam erat, sed diam voluptua. At vero eos et accusam et justo duo dolores et ea rebum. Stet clita kasd gubergren, no sea takimata sanctus est Lorem ipsum dolor sit amet.

