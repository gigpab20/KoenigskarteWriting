\Author{\daAuthorTwo}

\subsubsection{VS Code}
Visual Studio Code (VS Code) is a version of the Code-OSS project with Microsoft-specific customizations, making it a closed-source product licensed by Microsoft. VS Code is highly customizable, with support for community-made plugins and many options for the user to define how things should act and look. \autocite{VSCode:readMe} There are many handy features implemented out of the box, such as native VCS support. Since VS Code is an electron based application, it can not only be used on a desktop computer, but also directly in the browser of any device. \autocite{VSCode:electron} \autocite{VSCode:web}

\begin{figure} [H]
    \center
    \includegraphics [width=0.2\textwidth] {images/Technologies/vscodeLogo.png}
    \caption{VS Code Logo (Source: \url{https://code.visualstudio.com/brand})}
\end{figure}

\subsubsection{IntelliJ}
IntelliJ IDEA is an IDE primarily designed for Java and Kotlin developers, published by JetBrains. In comparison to VS Code, it is more resource-intensive due to included tools like the direct integration of a database connection via the embedded DataGrip version. IntelliJ is the base framework used for other, more specific IDEs by JetBrains, like Android Studio, WebStorm or DataGrip. Its functionality can also be extended through plugins. \autocite{IntelliJ:JetBrains} \autocite{IntelliJ:Wikipedia}

\begin{figure} [H]
    \center
    \includegraphics [width=0.4\textwidth] {images/Technologies/intellijLogo.png}
    \caption{IntelliJ IDEA Logo (Source: \url{https://www.jetbrains.com/company/brand/})}
\end{figure}

\subsubsection{Android Studio}
Android Studio is based on the IntelliJ framework, but, as the name suggests, with the specification to developing Android apps. Aside from Java and Kotlin based pure Android apps, it can also be used for developing cross-platform applications using the flutter framework. The IDE introduced by Google also features an Emulator for Pixel devices which simplifies testing. Lastly, it provides an intuitive way to compile the app and flash it to a physical Android device using ADB and USB-debugging. 

\begin{figure} [H]
    \center
    \includegraphics [width=0.3\textwidth] {images/Technologies/androidstudioLogo.png}
    \caption{Android Studio Logo (Source: \url{https://img.icons8.com/?size=100&id=04OFrkjznvcd&format=png&color=000000})}
\end{figure}

\subsubsection{Postman}
Postman is an API platform, primarily used to test APIs. It features collaboration and an automatic code-generation, allowing users to design requests directly in postman and export them into any of the supported languages. We used it to develop and test our API, without a frontend application to minimize potential points of failure.

\begin{figure} [H]
    \center
    \includegraphics [width=0.3\textwidth] {images/Technologies/postmanLogo.png}
    \caption{Postman Logo (Source: \url{https://www.postman.com/legal/logo-usage/})}
\end{figure}

\newpage