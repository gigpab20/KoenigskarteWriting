\Author{\daAuthorTwo}

\subsubsection{Git}
\begin{figure}[H]
    \centering
    \begin{minipage}{0.6\textwidth}
      \setlength{\baselineskip}{1.5em}
      \vspace{-1em}
      Git is a distributed VCS that was developed by Linus Torvalds in 2005. The main benefit of a VCS is to easily keep track of different versions of files. Git is the most widely used option concerning this area of application.  \autocite{Git:survey} This is due to the fact it is open-source, therefore free, as well as reliable and easy to learn. It gets used in almost every, but not only, development project—not just for tracking the history of files but also for developing cooperatively, making use of its branching feature. This allows for development while maintaining a stable version on the main branch. \autocite{Git:branching}
    \end{minipage}
    \hfill
    \begin{minipage}{0.35\textwidth}
        \center
    \includegraphics [width=1\textwidth] {images/Technologies/gitLogo.png}
    \caption{Git Logo (Source: \url{https://git-scm.com/})}
\end{minipage}
\end{figure}


\subsubsection{GitHub}
\begin{figure}[H]
    \centering
    \begin{minipage}{0.6\textwidth}
      \setlength{\baselineskip}{1.5em}
      \vspace{-1em}
      GitHub is a platform maintained by Microsoft. As the name implies, it is based on Git and provides the opportunity to easily share your Git repositories with other users. It is free to use for non-commercial applications and a very popular option when it comes to developing in a team. \autocite{GitHub:whatIsIt} Furthermore, it also implements handy extensions that integrate tightly with Git, for example GitHub-Actions, which is their solution for CI/CD pipelines. There are many alternatives to GitHub, like GitLab, which is open-source and can be self-hosted, or BitBucket, a solution by Atlassian. \autocite{GitHub:actions}
    \end{minipage}
    \hfill
    \begin{minipage}{0.35\textwidth}
        \center
    \includegraphics [width=1\textwidth] {images/Technologies/gitHubLogo.png}
    \caption{GitHub Logo (Source: \url{https://github.com/})}
\end{minipage}
\end{figure}